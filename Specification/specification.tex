%\section{Specs}
\section{Problem description}\label{sec:prob_desc}
Le regolamentazioni vigenti riguardo la conservazione dei generi alimentari impone ad ogni commerciante l'obbligo di mantenere aggiornato un registro giornaliero contenente le temperature delle celle frigorifere usate; in caso di mancato rispetto, il commerciante è tenuto a pagare una penale molto consistente.
\vspace{3mm}

Inoltre, è conveniente per il commerciante sapere in automatico varie informazioni sullo stato dei frigoriferi (temperatura, umidità etc...) senza il bisogno di andare a controllare personalmente il loro stato, inoltre è conveniente avere tali informazioni sempre alla propria portata tramite l'uso di smartphone.

\vspace{3mm}

Per questo scopo desideriamo introdurre nomeprogetto, come valida soluzione sia al singolo commerciante e ad imprese che hanno il bisogno di tenersi sempre aggiornate sullo stato delle proprie celle frigorifere e, di conseguenza, sui loro prodotti che potrebbero finire in pessimo stato nel caso qualcosa potesse malfunzionare.
%The advent of the COVID-19 pandemic has set limits regarding crowding in public places where a minimum social distance must be maintained to prevent the spread of the infection. This event has increased the amount of time (wait) required to carry out normal daily habits, such as shopping in stores, interacting with public offices, etc.
%L'avvento della pandemia causata dal COVID-19 ha posto dei limiti per quanto riguarda l'affollamento nei luoghi pubblici, nei quali occorre mantenere un distanziamento sociale minimo per prevenire la diffusione del contagio. Questo evento ha aumentato la quantità di tempo (attesa) necessaria per svolgere le normali abitudini giornaliere, ad esempio fare acquisti nei negozi, interagire con gli uffici pubblici, etc.

%Knowing in advance the amount of people in a given place could, without a doubt, helps in containing the virus and in optimizing the time lost in unnecessary waiting.
%Conoscere in anticipo la quantità di persone in un determinato luogo potrebbe, senza dubbio, aiutare nel contenimento del virus e contribuire ad ottimizzare il tempo perso in inutili attese.
\section{Use case(s)}
Come abbiamo menzionato in \ref{sec:prob_desc}, il nostro progetto ha lo scopo di fornire una soluzione efficiente e smart per la conservazione di prodotti in celle frigorifere, applicando elementi smart ed iot: per far ciò disporremo di sensori rilevanti le temperature ed umidità delle singole celle, tali dati saranno salvati al termine di ogni finestra temporale decisa dall'utente (la quale può variare anche da stagione a stagione) su un database apposito.
\vspace{3mm}

I clienti si potranno interfacciare tramite l'uso di un bot telegram che provvederà ad avvertire gli utenti nel caso ci fossero problemi urgenti alle proprie celle o per fornire i dati giornalieri al termine dell'attività lavorativa (anche questo parametro è impostabile dall'utente). Ogni sensore avrà un identificatore univoco che li contraddistinguerà che sarà salvato su un altro database in cui saranno contenute anche le specifiche desiderate dai clienti.
\vspace{3mm}

Con quanto abbiamo descritto brevemente in questo paragrafo sarà possibile mantenere aggionrnati gli utenti nel caso di guasti improvvisi e/o altre problematiche legate all'utilizzo delle celle frigorifere, oltre alla possiblità di ottenere in modo repentino dei dati fondamentali da riportare sul registro giornaliero.
%As we mentioned in \ref{sec:prob_desc}, a possible application of our project could take place through the use of a bot, to which it is possible to ask directly the level of crowding of a particular place. The latter must be equipped with cameras positioned in strategic places, like entrances and exits. For this purpose it is also possible to use the surveillance cameras already present in the shops/offices.
%Come abbiamo accennato in \ref{sec:prob_desc}, una possibile applicazione del nostro progetto potrebbe avvenire mediante l'utilizzo di un bot, a cui è possibile chiedere direttamente il livello di affluenza di un determinato luogo. Quest'ultimo deve essere dotato di telecamere posizionate in luoghi strategici, come lo sono le entrate e le uscite. Per tale scopo è anche possibile utilizzare le telecamere di sorveglianza già presenti nei locali.

%A further use case could be meeting the merchant's or public body's need to know the number of customers over a given period of time.

%Un ulteriore caso d'uso potrebbe essere il soddisfacimento del bisogno del commerciante od ente pubblico del sapere il numero di clienti in un determinato periodo di tempo.

\section{Requirements}
%Andremo ad utilizzare le seguenti piattaforme hardware/software:
We'll use the following software/hardware platforms:
\begin{itemize}
    \item \textbf{Hardware} side:
    \begin{itemize}
        \item \textbf{IP camera} with Wi-Fi module for video streaming;%con modulo Wi-Fi per il flusso video;
        \item \textbf{computer} to run the software.%Macchina su cui far girare il software.
    \end{itemize}
    \item \textbf{Software} side:
    \begin{itemize}
        \item \textbf{telegram bot}, built on Python by using python-telegram-bot;
        \item \textbf{software for people detection} with the use of OpenCV and for sending data towards the ThingSpeak platform;%con l'uso di OpenCV, e per l'invio di informazioni diretto alla piattaforma ThingSpeak.
        \item \textbf{Heroku} for bot hosting;%per l'hosting del bot;
        \item \textbf{Thingspeak} for the online data storing and visualization;%il servizio di mantenimento online e visualizzazione dei dati;
        \item \textbf{MongoDB} for the storing of data regarding the affluence of people inside a building.%per il salvataggio dei dati riguardanti l'affluenza delle persone in un determinato locale.
    \end{itemize}
    %\item Bot:
\end{itemize}

\section{Solution description}
Lo scopo del nostro progetto è fornire una soluzione iot ai commercianti e/o imprese che hanno il bisogno di essere sempre aggiornate sullo stato dei loro frigoriferi; per far fronte a ciò abbiamo intenzione di produrre un bot telegram che (a seguito di alcuni parametri definiti dall'utente) aggiorni il cliente sulle condizioni delle celle senza il bisogno di recarsi in loco.
\vspace{3mm}

Il nostro progetto sarà in scala minore rispetto a quanto ci attendiamo, di fatto utilizzeremo un singolo raspberry pi4 con tre sensori di temperatura annessi in modo da monitorare lo stato delle celle. I sensori saranno distinti da un codice identificativo univoco preceduto dal nome/codice dell'impresa che lo possiede (in questo modo possiamo sapere a chi fa riferimento quel sensore ed in quale delle celle si trova); i dati (temperatura ed umidità) saranno salvati, periodicamente su scelta dell'utente, su un server necessario per il loro immagazzinamento e visualizzazione. A questo punto i dati sono disponibili per il bot, ospitato sulla piattaforma Heroku, il quale potrà farne uso per i propri scopi.
%The goal of our project is to produce a bot that, at the request of the user, indicates the number of people in a room. At this point we will use IP cameras with the Wi-Fi module in order to simulate the behavior of a real camera. When the camera is turned on, the software starts counting in real time, through the OpenCV library, the number of people who are currently inside the room.

%L'obiettivo del nostro progetto è quello di produrre un bot che, su richiesta dell'utente, indichi il numero di persone all'interno di un locale. A tal punto utilizzeremo delle IP camera con il modulo Wi-Fi al fine di simulare il comportamento di una vera telecamera. All'accensione della telecamera, il software inizia a contare in tempo reale, attraverso la libreria OpenCV, il numero di persone che attualmente sono all'interno del locale.

%The software also, at each time slot decided on the basis of the local (arbitrarily in this project, for simplicity), saves on the database the data on the turnout that will be recoverable by the bot, which will provide this information at the user's request.

%Il software inoltre, ad ogni slot temporale deciso in base al locale (arbitrariamente in questo progetto, per semplicità), salva sul database il dato sull'affluenza che sarà recuperabile dal bot, il quale fornirà tale informazione a richiesta dell'utente. 

\section{Planned demo and future work}
Presenteremo il nostro progetto tenendo in considerazione il suo fattore di scalabilità per un suo possibile impiego maggiore in futuro, ma per rimanere in termini strettamente accedemici e per via del tempo a noi concessoci per la realizzazione, mostreremo una piccola implementazione del nostro lavoro.
\vspace{3mm}

La presentazione del nostro lavoro avverrà attraverso un video e/o una piccola dimostrazione su come fare il setup del bot per un singolo utente; nel caso decidessimo di mostrare il video, faremo il possibile per elencare ogni singolo passaggio per il corretto funzionamento del bot e del ruolo dei sensori nel nostro progetto.
\vspace{3mm}

Siamo estremamente interessati ad espandere il nostro lavoro con la possibilità di inviare sms in parallelo col bot, in modo da rendere tutto più accessibile anche alle fasce più anziane dei clienti oltre alla creazione di un'app che possa dare maggiore capacità di gestione sui sensori da parte degli utenti.
%We will present a demonstration video of our work, in which we will show how people are detected (and how the affluence value on the cloud is updated), and then show a possible use of the bot.

%Presenteremo un video dimostrativo del nostro lavoro, nel quale mostreremo come avviene il rilevamento delle persone (e come viene aggiornato il valore di affluenza sul cloud), per poi presentare un possibile utilizzo del bot.

%In the future we intend to expand our work through the possibility of creating a real app in which it will be possible to make a possible reservation to a shop/office.
%We are also interested in expanding our project by collecting further data for statistical purposes to better understand the flow of people in a certain area and/or at a certain time.

%In futuro avevamo intenzione di espandere il nostro lavoro attraverso la possibilità di creazione di un'app vera e propria con la quale sia possibile effettuare una possibili prenotazione verso un locale.
%Siamo inoltre interessati nell'ampliare il nostro progetto con la rilevazione di ulteriori dati a fini statistici per comprendere il flusso di persone in una determinata zona e/o in un determinato orario.

Take a training example (a set of six movie preferences). Set the states of the visible units to these preferences.
Next, update the states of the hidden units using the logistic activation rule described above: for the $j$th hidden unit, compute its activation energy $a_j = \sum_i w_{ij} x_i$, and set $x_j$ to 1 with probability $\sigma(a_j)$ and to 0 with probability $1 - \sigma(a_j)$. Then for each edge $e_{ij}$, compute $Positive(e_{ij}) = x_i * x_j$ (i.e., for each pair of units, measure whether they're both on).
Now reconstruct the visible units in a similar manner: for each visible unit, compute its activation energy $a_i$, and update its state. (Note that this reconstruction may not match the original preferences.) Then update the hidden units again, and compute $Negative(e_{ij}) = x_i * x_j$ for each edge.
Update the weight of each edge $e_{ij}$ by setting $w_{ij} = w_{ij} + L * (Positive(e_{ij}) - Negative(e_{ij}))$, where $L$ is a learning rate.
Repeat over all training examples.